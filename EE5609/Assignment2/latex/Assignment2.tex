\documentclass[journal,12pt,twocolumn]{IEEEtran}
\usepackage[utf8]{inputenc}
\usepackage{amsmath}
\usepackage{mathtools}
\usepackage{txfonts}
\usepackage{listings}
\usepackage{hyperref}
\usepackage{multirow}

\renewcommand\thesection{\arabic{section}}
\renewcommand\thesubsection{\thesection.\arabic{subsection}}
\renewcommand\thesubsubsection{\thesubsection.\arabic{subsubsection}}

\renewcommand\thesectiondis{\arabic{section}}
\renewcommand\thesubsectiondis{\thesectiondis.\arabic{subsection}}
\renewcommand\thesubsubsectiondis{\thesubsectiondis.\arabic{subsubsection}}

\lstset{
%language=C,
frame=single, 
breaklines=true,
columns=fullflexible
}

\def\putbox#1#2#3{\makebox[0in][l]{\makebox[#1][l]{}\raisebox{\baselineskip}[0in][0in]{\raisebox{#2}[0in][0in]{#3}}}}
     \def\rightbox#1{\makebox[0in][r]{#1}}
     \def\centbox#1{\makebox[0in]{#1}}
     \def\topbox#1{\raisebox{-\baselineskip}[0in][0in]{#1}}
     \def\midbox#1{\raisebox{-0.5\baselineskip}[0in][0in]{#1}}
\vspace{3cm}

\title{ASSIGNMENT 2}
\author{Anjana V\\AI20MTECH14010 }

\begin{document}
\numberwithin{equation}{subsection}
\makeatletter
\renewcommand*\env@matrix[1][*\c@MaxMatrixCols c]{%
  \hskip -\arraycolsep
  \let\@ifnextchar\new@ifnextchar
  \array{#1}}
\makeatother
\newcommand{\myvec}[1]{\ensuremath{\begin{pmatrix}#1\end{pmatrix}}}
\newcommand{\BAR}{%
  \hspace{-\arraycolsep}%
  \strut\vrule % the `\vrule` is as high and deep as a strut
  \hspace{-\arraycolsep}%
}
\renewcommand{\vec}[1]{\mathbf{#1}}
\maketitle
\begin{abstract}
This document examines the consistency of equations.
\end{abstract}
Download all python codes from 
%
\begin{lstlisting}
https://github.com/anjanavasudevan/grad_schoolwork/tree/master/EE5609/Assignment2
\end{lstlisting}
%
and latex-tikz codes from 
%
\begin{lstlisting}
https://github.com/anjanavasudevan/grad_schoolwork/tree/master/EE5609/Assignment2/latex
\end{lstlisting}
%
\section{Question No. 55}
Examine the consistency of the system of given equations:
\begin{align}
		x+y+z =1 
		\\2x+3y+2z = 2
		\\ax+ay+2az = 4
\end{align}
\section{Answer}
Assume that $a$ is any real number. The above system of equations can be expressed in the form of matrix:
\begin{align}
\myvec{1 & 1 & 1\\2 & 3 & 2\\a & a & 2a} \myvec{x\\y\\z} = \myvec{1\\2\\4}
\end{align}
This is in the form of:
\begin{align}
\label{eqn_aug}
\vec{Ax}=\vec{B}
\end{align}
The system defined above has a solution only when
\begin{align*}
rank(\vec{A|B}) = rank(\vec{A}) = dim(\vec{A})
\end{align*}
Reducing the augmented matrix to row echelon form, we get:
\begin{align}
\myvec{1 & 1 & 1 & \BAR & 1 \\2 & 3 & 2 & \BAR & 2 \\a & a & 2a & \BAR & 4}
\xleftrightarrow[]{R_2\rightarrow R_2-2R_1}
\myvec{1 & 1 & 1 & \BAR & 1 \\0 & 1 & 0 & \BAR & 0 \\a & a & 2a & \BAR & 4}
\\ \myvec{1 & 1 & 1 & \BAR & 1 \\0 & 1 & 0 & \BAR & 0 \\a & a & 2a & \BAR & 4}\xleftrightarrow[]{R_3\rightarrow R_3-aR_1}
\myvec{1 & 1 & 1 & \BAR & 1 \\0 & 1 & 0 & \BAR & 0 \\0 & 0 & a & \BAR & 4-a}
\\ \implies rank(\vec{A}) = rank(\vec{A|B}) = dim(\vec{A})
\end{align}
The system of equations is consistent and has a unique solution except at $\vec{a = 0}$.

Back-solving the above set of equations, we get:
\begin{align}
%z = \frac{4-a}{a}
%\\y = 0
%\\x = \frac{2(a - 2)}{a}\\
\vec{x} = \myvec{\frac{2(a - 2)}{a} \\ 0 \\ \frac{4-a}{a}}
\end{align}
\end{document}